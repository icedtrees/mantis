\documentclass[]{article}
\usepackage{amsmath}
\usepackage{amssymb}
\usepackage{listings}
\usepackage{color}
\usepackage{graphicx}

\definecolor{dkgreen}{rgb}{0,0.6,0}
\definecolor{gray}{rgb}{0.5,0.5,0.5}
\definecolor{mauve}{rgb}{0.58,0,0.82}

\lstset{frame=tb,
  language=Python,
  frame=none,
  basicstyle={\small\ttfamily},
  numbers=none,
  keywordstyle=\color{blue},
  commentstyle=\color{dkgreen},
  breaklines=true,
  breakatwhitespace=true,
  tabsize=4
}

%opening
\title{A Teleoperated Command Library for a \\ Six Wheeled Rescue Robot \\ COMP3431}
\author{Chan G., Huang L., Mao D.}
\date{\today}

\begin{document}
\maketitle % Insert the title, author and date

\begin{abstract}
Current rescue robots are incredibly expensive and not suitable for student use. On the other hand DIY kits have many limitations and are not sufficiently customiseable for undergraduate or graduate use. The Open Academic Robot Kit aims to change that, however there is no library for the robot design that is available. In this paper a basic library is detailed, with locomotion, networking and compatibility with ROS.
\end{abstract}

\section{}
Disaster, such as the Great Hanshi-Awaji earthquake on the 17th of January 1995, or the advent of the Sept 11 attacks on the World Trade Centre necessarily impel research into better search and rescue strategies, and search and rescue robots (Kobe University Library, 1999; Davids, 2002). Rescue robots had been proved in the site of such catastrophe, and large amounts of funding have been invested. Nevertheless cost is still a significant consideration, with specialised robots running upwards of \$60,000 and as high as \$300,000 (Bigge, 2011). Certainly not all rescue robots are so expensive, but price is still rather inhibitory to hobbyists and students, who have much to add to the rescue robotic endeavour. On the other hand robotics kits geared towards student use are limited in function, difficult to customise, and have poor compatibility with parts from other companies or outside sources (Open Academic Robotic Kit, 2014).
\\
\\
With this in mind, cost effective models with a high degree of flexibility were looked into. Sheh has developed a 3D printed open source design for a teleoperated search and rescue robot as part of the Open Academic Robotic Kit (OARKit) (Sheh, R. 2014). The complete kit is expected to cost around \$500, which is well within the budget of schools and universities, however it has no source code as yet. The aim of the project then, is to develop a basic library that can be used in conjunction with OARKit, in order to facilitate use as a teaching aid.

\section{Literature Review}

The robot has been designed for compatibility with Dynamixel AX-12 servo motors. Several libraries currently exist for controlling the motors. Libraries written by Pablo Gindel, Savage Electronics and Scott Fergusson were looked into.
\\
\\
Pablo Gindel has developed a C++ library for controlling the motors with an Arduino (Gindel, 2010). Ideally the number of components would be minimised to minimise cost and complexity. In order to achieve this, it was decided that preferably a Raspberry Pi would directly interface with the servo motors and an Arduino would not be used. Additionally Gindel's library had several bugs and documentation was not written in English.
\\
\\
Similarly the Savage Electronics library was written in C++. It also is designed to work with an Arduino, but also requires additional hardware. Again, the documentation is not English.
\\
\\
ForestMoon Dynamixel library by Scott Ferguson was originally written in C\#, which posed a problem as ROS is currently only implemented in C++, python and Java (Ferguson). Fortunately Patrick Goebel wrote a python version (Goebel, 2014). Ferguson's library is able to be used on a Raspberry Pi, and documentation is English, so it was chosen over the other options.
\\
TODO: discuss bug fixes done?
\\
\\
There are several main methods of steering, each with their own advantages and disadvantages (Shamah et. al., 2001). By virtue of chassis design a range of steering options were unable to be implemented. These included ackerman, independent, synchronous or omnidirectional steering. The remaining options were skid steering and articulated drive.
\\
\\
Skid steering has very high maneuverability for low mechanical complexity. If skid steering alone is implemented, a lot of space is saved because of the low number of components. However because of the differential thrust from the right and left there is relatively poor traction, and there is relatively high wear on the wheels. This leads to a relatively high power consumption. (Kang et. al., 2010)
\\
\\
Conversely articulated drive has a significantly larger turning circle, and  maintains traction throughout the turn, allowing for acceleration throughout. Implementation is more involved than skid steering however, as outer wheels have to be turning faster than inner wheels. Shamah et. al. (2001) developed a series of velocity calculation equations for a four wheeled vehicle with an articulated axle. The velocity of the individual wheel was calculated based on the distance from each individual wheel to the centre of the circle, and the desired robot velocity. (See fig 1.)
\\
\\
As a library was being made for use with the OARKit, both skid steering and articulated steering were implemented. This would allow the end user to choose the most suitable method for their environment and design specification.
\\
\\
\begin{figure}
\includegraphics[width=\textwidth]{fig1}
\caption{(Shamah et. al. 2001)}
\end{figure}
Used ROS joystick driver to teleop and testing and blah (ref).


\section{References}
Bigge, R. (2011). \textit{Robots to the rescue}. Available at\\
https://secure.globeadvisor.com/servlet/ArticleNews/story/gam/20110527\\
/ROBMAG\_JUNE2011\_P12\_14\_ (accessed 04/11/2014).\\
\\
Davids, A. (2002). Urban search and rescue robots: from tragedy to technology. \textit{Intelligent Systems, IEEE 17}(2). 81 - 83.\\
\\
Gindel, P. (2010). \textit{Arduino library for AX-12}. \\Available at http://www.pablogindel.com/2010/01/biblioteca-de-arduino-para-ax-12/. (accessed 09/11/14).\\
\\
Kang, J., Kim W., Lee, J., \& Yi, K. (2010). Design, implementation, and test of skid steering-based autonomous driving controller for a robotic vehicle with articulated suspension. \textit{Journal of Mechanical Science and Technology 24}(3). 793 - 800.\\
\\
Kobe University Library. (1999) \textit{Great Hanshin-Awaji Earthquake Disaster Materials Collection.} Available at: http://www.lib.kobe-u.ac.jp/eqb/e-gallery.html (accessed 04/11/2014).\\
\\
The Open Academic Robotic Kit. Available at www.oarkit.org (accessed 09/11/2014).\\
\\
Shamah, B.,  Wagner, M. D., Moorehead S., Teza, J., Wettergreen, D., \& Whittaker, W. (2001). \textit{Steering and Control of a Passively Articulated Robot}. The Field Robotics Center, Carnegie Mellon University.\\
\\
Sheh, R. (2014). \textit{Open Academic Robot Kit: The Six-Wheeled Wonder - a 6 Wheel Drive robot platform using Dynamixel AX-12A servos}. Available at: http://www.thingiverse.com/thing:327689 (accessed 04/11/2014).

\end{document}

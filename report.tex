\documentclass[]{article}
\usepackage{amsmath}
\usepackage{amssymb}
\usepackage{listings}
\usepackage{color}
\usepackage{graphicx}

\definecolor{dkgreen}{rgb}{0,0.6,0}
\definecolor{gray}{rgb}{0.5,0.5,0.5}
\definecolor{mauve}{rgb}{0.58,0,0.82}

\lstset{frame=tb,
  language=Python,
  frame=none,
  basicstyle={\small\ttfamily},
  numbers=none,
  keywordstyle=\color{blue},
  commentstyle=\color{dkgreen},
  breaklines=true,
  breakatwhitespace=true,
  tabsize=4
}

%opening
\title{A Teleoperated Command Library for a \\ Six Wheeled Rescue Robot \\ COMP3431}
\author{Chan G., Huang L., Mao D.}
\date{\today}

\begin{document}
\maketitle % Insert the title, author and date

\begin{abstract}
Current rescue robots are incredibly expensive and not suitable for student use. On the other hand DIY kits have many limitations and are not sufficiently customiseable for undergraduate or graduate use. The Open Academic Robot Kit aims to change that, however there is no library for the robot design that is available. In this paper a basic library is detailed, with locomotion, networking and compatibility with ROS.
\end{abstract}

\section{}
Disaster, such as the Great Hanshi-Awaji earthquake on the 17th of January 1995, or the advent of the Sept 11 attacks on the World Trade Centre necessarily impel research into better search and rescue strategies, and search and rescue robots (Kobe University Library, 1999; Davids, 2002). Rescue robots had been proved in the site of such catastrophe, and large amounts of funding have been invested. Nevertheless cost is still a significant consideration, with specialised robots running upwards of \$60,000 and as high as \$300,000 (Bigge, 2011). Certainly not all rescue robots are so expensive, but price is still rather inhibitory to hobbyists and students, who have much to add to the rescue robotic endeavour. On the other hand robotics kits geared towards student use are limited in function, difficult to customise, and have poor compatibility with parts from other companies or outside sources (Open Academic Robotic Kit, 2014).
\\
\\
With this in mind, cost effective models with a high degree of flexibility were looked into. Sheh has developed a 3D printed open source design for a teleoperated search and rescue robot as part of the Open Academic Robotic Kit (OARKit) (Sheh, R. 2014). The complete kit is expected to cost around \$500, which is well within the budget of schools and universities, however it has no source code as yet. The aim of the project then, is to develop a basic library that can be used in conjunction with OARKit, in order to facilitate use as a teaching aid.

\section{Literature Review}

The robot has been designed for compatibility with Dynamixel AX-12 servo motors. Several libraries currently exist for controlling the motors. Libraries written by Pablo Gindel, Savage Electronics and Scott Fergusson were looked into.
\\
\\
Pablo Gindel has developed a C++ library for controlling the motors with an Arduino (Gindel, 2010). Ideally the number of components would be minimised to minimise cost and complexity. In order to achieve this, it was decided that a Raspberry Pi would directly interface with the servo motors and an Arduino would not be used. Additionally Gindel's library had several bugs and documentation was not written in English.
\\
\\
\textbf{REMOVE THIS
Choice of library to control the dynamixel ax-12. Pablo Gindel - slightly outdated, with minor bugs, not english. Savage Electronics, again not english, requires extra hardware (ref, ref). Used pydynamixel by Scott Fergusson (ref). python cause easier to use with ROS.  Had to fix bugs}
\\
\\
There are several main methods of steering, each with their own advantages and disadvantages (Shamah et. al., 2001). By virtue of chassis design a range of steering options were unable to be implemented. These included ackerman, independent, synchronous or omnidirectional steering. The remaining options were skid steering and articulated drive.
\\
\\
Skid steering has very high maneuverability for low mechanical complexity. If skid steering alone is implemented, a lot of space is saved because of the low number of components. However because of the differential thrust from the right and left there is relatively poor traction, and there is relatively high wear on the wheels. This leads to a relatively high power consumption. (Kang et. al., 2010)
\\
\\
Conversely articulated drive has a significantly larger turning circle, and  maintains traction throughout the turn, allowing for acceleration throughout. Implementation is more involved than skid steering however, as outer wheels have to be turning faster than inner wheels. Shamah et. al. (2001) developed a series of velocity calculation equations for a four wheeled vehicle with an articulated axle. The velocity of the individual wheel was calculated based on the distance from each individual wheel to the centre of the circle, and the desired robot velocity. (See fig 1.)
\\
\\
As a library was being made for use with the OARKit, both skid steering and articulated steering were implemented. This would allow the end user to choose the most suitable method for their environment and design specification.
\\
\\
\begin{figure}
\includegraphics[width=\textwidth]{fig1}
\caption{(Shamah et. al. 2001)}
\end{figure}
\\
\\
Used ROS joystick driver to teleop and testing and blah (ref).

\section{Locomotion}
The OARKit robot has a fairly rare design in which six wheels can turn relative to each other via two rotating joints, each with a range of about $\pm20$ degrees.
\\
\\
figure goes here
\\
\\
There have been similar studies into wheel velocities for similar vehicles, such as Shamah et al. (2001)'s calculations diagrammed above, but the OARKit robot is unique in that it has mobile joints rather than rotating axles, and it also has six wheels for better locomotion.
\\
\\
We use similar calculations to Shamah et al., by definining a common intersection point for lines passing through all the axles, and using the distance between the wheel and the common intersection point to scale the relative velocity of the wheel.
\\
\\
diagram goes here
\\
\\
Since we have six wheels and not four, there are six different axles. Thus we must make sure that the back axle is angled such that the line extended inwards would intersect with the intersection of the front two axle lines.
\\
\\
For $\theta_{front} > \frac{\pi}{2}$:
\\
$\theta_{back} = tan^{-1}(tan(\theta_{front}) \times \frac{spine_{top}}{spine_{bottom}})$
\\
\\
For $\theta_{front} > \frac{\pi}{2}$:
\\
$\theta_{back} = \pi - tan^{-1}(tan(\pi - \theta_{front}) \times \frac{spine_{top}}{spine_{bottom}})$
\\
\\
With the angle values of the top and bottom, we can thus define the relative velocities of all the wheels as the distances from the center point:
\\
\[ \omega_{topleft} \propto \frac{1}{sin(a)} + \frac{arm_{topleft}}{spine_{top} * tan(\theta_{top})} \]
\[ \omega_{topright} \propto \frac{1}{sin(1)} - \frac{arm_{topright}}{spine_{top} * tan(\theta_{top})} \]
\[ \omega_{middleleft} \propto 1 + \frac{arm_{midleft}}{spine_{top} * tan(\theta_{top})} \]
\[ \omega_{middleleft} \propto 1 - \frac{arm_{midright}}{spine_{top} * tan(\theta_{top})} \]
\[ \omega_{bottomleft} \propto \frac{spine_{bottom}}{spine_{top} * tan(\theta_{top}) * cos(\theta{bottom})} + \frac{arm_{bottomleft}}{spine_{top} * tan(\theta_{top})} \]
\[ \omega_{bottomleft} \propto \frac{spine_{bottom}}{spine_{top} * tan(\theta_{top}) * cos(\theta{bottom})} - \frac{arm_{bottomright}}{spine_{top} * tan(\theta_{top})} \]
\\
\\
We use this kind of articulated steering as the baseline velocity for all the wheels when the robot is at a certain inclination. Ideally this would cause all wheels to travel exactly at the correct tangential velocity.
\\
In our mantis library, we define the velocity of the vehicle $v$ as the tangential velocity of the center point of the middle axle (which is $spine_{top} * tan(\theta_{top})$ distance from the central intersection). We can multiply any of the above formulae by $v$ to obtain a velocity relative to the velocity of the center of the vehicle:
\\
\[ \omega_{topleft} = v * (\frac{1}{sin(a)} + \frac{arm_{topleft}}{spine_{top} * tan(\theta_{top})}) \]
\[ \omega_{topright} = v * (\frac{1}{sin(1)} - \frac{arm_{topright}}{spine_{top} * tan(\theta_{top})}) \]
\[ \omega_{middleleft} = v * (1 + \frac{arm_{midleft}}{spine_{top} * tan(\theta_{top})}) \]
\[ \omega_{middleleft} = v * (1 - \frac{arm_{midright}}{spine_{top} * tan(\theta_{top})}) \]
\[ \omega_{bottomleft} = v * (\frac{spine_{bottom}}{spine_{top} * tan(\theta_{top}) * cos(\theta{bottom})} + \frac{arm_{bottomleft}}{spine_{top} * tan(\theta_{top})}) \]
\[ \omega_{bottomleft} = v * (\frac{spine_{bottom}}{spine_{top} * tan(\theta_{top}) * cos(\theta{bottom})} - \frac{arm_{bottomright}}{spine_{top} * tan(\theta_{top})}) \]
\\
This method of locomotion is quite efficient - the wheels are all synchronised fairly well, and there is little drag of the tyres on the ground. However, one can also add a skid steering velocity onto the wheels on each side, sacrificing efficiency to be able to perform a tighter turn.
\\
We implement this additively - there is a skid steering command available that will add a certain velocity to the wheels on each side of the robot, aiding turns greatly. For example, if a velocity of $2pi m/s$ was added to one side, and a velocity of $-2pi m/s$ was added to the other side, the vehicle would theoretically rotate at approximately $1 / radius$ revolutions per second along its current course, something not achievable with articulated steering. However, as mentioned before, since this is being performed on six wheels, there is a fair amount of drag between the wheels and the ground. This can be reduced by lifting up the front portion of the robot, so only four wheels are in contact with the ground.

\subsection{Results}
During our tests of the robot, we found that both methods of steering were effective on wood and carpet surfaces. The six-wheeled skid steering had some issues on the carpet because the wheels gripped the surface too hard, but if the front portion of the robot was lifted up, there were no resulting problems. Additionally, the skid steering was unreliable and caused the robot to rotate differing angular distances on different surfaces.
\\
However, the robot did also have the problem of not having enough grip. The wheels were constructed from ABS filament and had no coating, which means they had very little friction with the wooden surface and had some trouble performing maneuvers on sloped wooden surfaces, as it would slip down.
\\
It was also noted that it was quite possible to reliably execute tight turns on multiple surfaces by using common vehicle maneuvers, such as three-point turns, in conjunction with articulated steering.

\subsection{Interface}
We provide several interface functions in mantis.py that allow control of the robot with the designated steering system:
\begin{itemize}
    \item move(int -1023 to 1023): Move the robot at a given velocity (positive or negative). Path and direction is determined by current articulation.
    \item skid(int -1023 to 1023): Add a skid steering velocity to the current movement.
    \item turn\_to(int -1 to 1): Set the articulation to a given value between -1 (minimum articulation) and 1 (maximum articulation)
    \item lift\_to(int -1 to 1): Set the lift of the front segment to a given value between -1 and 1.
    \item turn(int -1023 to 1023): A convenience function that sets the robot to change its articulation at a certain speed until it reaches the maximum or minimum articulation.
    \item lift(int -1023 to 1023): Same as above, except with the lifting joint.
\end{itemize}

We define multiple constants which refer to all the dimensions and maximum and minimum turning points for the servomotors. These can be found in pyoarkit/config.py

\section{References}
Bigge, R. (2011). \textit{Robots to the rescue}. Available at\\
https://secure.globeadvisor.com/servlet/ArticleNews/story/gam/20110527\\
/ROBMAG\_JUNE2011\_P12\_14\_ (accessed 04/11/2014).\\
\\
Davids, A. (2002). Urban search and rescue robots: from tragedy to technology. \textit{Intelligent Systems, IEEE 17}(2). 81 - 83.\\
\\
Gindel, P. (2010). \textit{Arduino library for AX-12}. \\Available at http://www.pablogindel.com/2010/01/biblioteca-de-arduino-para-ax-12/. (accessed 09/11/14).\\
\\
Kang, J., Kim W., Lee, J., \& Yi, K. (2010). Design, implementation, and test of skid steering-based autonomous driving controller for a robotic vehicle with articulated suspension. \textit{Journal of Mechanical Science and Technology 24}(3). 793 - 800.\\
\\
Kobe University Library. (1999) \textit{Great Hanshin-Awaji Earthquake Disaster Materials Collection.} Available at: http://www.lib.kobe-u.ac.jp/eqb/e-gallery.html (accessed 04/11/2014).\\
\\
The Open Academic Robotic Kit. Available at www.oarkit.org (accessed 09/11/2014).\\
\\
Shamah, B.,  Wagner, M. D., Moorehead S., Teza, J., Wettergreen, D., \& Whittaker, W. (2001). \textit{Steering and Control of a Passively Articulated Robot}. The Field Robotics Center, Carnegie Mellon University.\\
\\
Sheh, R. (2014). \textit{Open Academic Robot Kit: The Six-Wheeled Wonder - a 6 Wheel Drive robot platform using Dynamixel AX-12A servos}. Available at: http://www.thingiverse.com/thing:327689 (accessed 04/11/2014).

\end{document}

\documentclass{article}
\usepackage{amsmath}
\usepackage{amssymb}
\usepackage{listings}
\usepackage{color}
\usepackage{graphicx}

\setlength\parindent{0pt} % Removes all indentation from paragraphs

\definecolor{dkgreen}{rgb}{0,0.6,0}
\definecolor{gray}{rgb}{0.5,0.5,0.5}
\definecolor{mauve}{rgb}{0.58,0,0.82}

\lstset{frame=tb,
  language=Python,
  frame=none,
  basicstyle={\small\ttfamily},
  numbers=none,
  keywordstyle=\color{blue},
  commentstyle=\color{dkgreen},
  breaklines=true,
  breakatwhitespace=true,
  tabsize=4
}

\title{A Teleoperated Comand Library for a \\ Six Wheeled Rescue Robot \\ COMP3431}
\author{Chan G., Huang L., Mao D.}
\date{\today}

\begin{document}
\maketitle % Insert the title, author and date

\section{Introduction}
Disaster, such as the Great Hanshi-Awaji earthquake on the 17th of January 1995, or the advent of the Sept 11 attacks on the World Trade Centre necessarily impel research into better search and rescue strategies, and search and rescue robots (Kobe University Library, 1999; Davids, 2002). Rescue robots had been proved in the site of such catastrophe, and large amounts of funding have been invested. Nevertheless cost is still a significant consideration, with specialised robots running upwards of \$60,000 and as high as \$300,000 (Bigge, 2011). Certainly not all rescue robots are so expensive, but price is still rather inhibitory to hobbyists and students, who have much to add to the rescue robotic endeavour. On the other hand robotics kits geared towards student use are limited in function, difficult to customise, and have poor compatibility with parts from other companies or outside sources (Open Academic Robotic Kit, 2014).
\\
\\
With this in mind, cost effective models with a high degree of flexibility were looked into. Sheh has developed a 3D printed open source design for a teleoperated search and rescue robot as part of the Open Academic Robotic Kit (OARKit) (Sheh, R. 2014). The complete kit is expected to cost around \$500, which is well within the budget of schools and universities, however it has no source code as yet. The aim of the project then, is to develop a basic library that can be used in conjunction with OARKit, in order to facilitate use as a teaching aid.

\end{document}
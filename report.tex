\documentclass[]{article}
\usepackage{amsmath}
\usepackage{amssymb}
\usepackage{listings}
\usepackage{color}
\usepackage{graphicx}
\usepackage[T1]{fontenc}

\definecolor{dkgreen}{rgb}{0,0.6,0}
\definecolor{gray}{rgb}{0.5,0.5,0.5}
\definecolor{mauve}{rgb}{0.58,0,0.82}

\lstset{frame=tb,
  frame=none,
  basicstyle={\small\ttfamily},
  numbers=none,
  keywordstyle=\color{blue},
  commentstyle=\color{dkgreen},
  breaklines=true,
  breakatwhitespace=true,
  tabsize=4
}

%opening
\title{A Teleoperated Command Library for a \\ Six Wheeled Rescue Robot \\ COMP3431}
\author{Chan G., Huang L., Mao D.}
\date{\today}

\begin{document}
\maketitle % Insert the title, author and date

\begin{abstract}
Current rescue robots are incredibly expensive and not suitable for student use. On the other hand DIY kits have many limitations and are not sufficiently customiseable for undergraduate or graduate use. The Open Academic Robot Kit aims to change that, however there is no library for the robot design that is available. In this paper a basic library is detailed, with locomotion, networking and compatibility with ROS.
\end{abstract}

\section{}
Disaster, such as the Great Hanshi-Awaji earthquake on the 17th of January 1995, or the advent of the Sept 11 attacks on the World Trade Centre necessarily impel research into better search and rescue strategies, and search and rescue robots (Kobe University Library, 1999; Davids, 2002). Rescue robots had been proved in the site of such catastrophe, and large amounts of funding have been invested. Nevertheless cost is still a significant consideration, with specialised robots running upwards of \$60,000 and as high as \$300,000 (Bigge, 2011). Certainly not all rescue robots are so expensive, but price is still rather inhibitory to hobbyists and students, who have much to add to the rescue robotic endeavour. On the other hand robotics kits geared towards student use are limited in function, difficult to customise, and have poor compatibility with parts from other companies or outside sources (Open Academic Robotic Kit, 2014).
\\
\\
With this in mind, cost effective models with a high degree of flexibility were looked into. Sheh has developed a 3D printed open source design for a teleoperated search and rescue robot as part of the Open Academic Robotic Kit (OARKit) (Sheh, R. 2014). The complete kit is expected to cost around \$500, which is well within the budget of schools and universities, however it has no source code as yet. The aim of the project then, is to develop a basic library that can be used in conjunction with OARKit, in order to facilitate use as a teaching aid.

\section{Literature Review}

The robot has been designed for compatibility with Dynamixel AX-12 servo motors. Several libraries currently exist for controlling the motors. Libraries written by Pablo Gindel, Savage Electronics and Scott Fergusson were looked into.
\\
\\
Pablo Gindel has developed a C++ library for controlling the motors with an Arduino (Gindel, 2010). Ideally the number of components would be minimised to minimise cost and complexity. In order to achieve this, it was decided that preferably a Raspberry Pi would directly interface with the servo motors and an Arduino would not be used. Additionally Gindel's library had several bugs and documentation was not written in English.
\\
\\
Similarly the Savage Electronics library was written in C++. It also is designed to work with an Arduino, but also requires additional hardware. Again, the documentation is not English.
\\
\\
ForestMoon Dynamixel library by Scott Ferguson was originally written in C\#, which posed a problem as ROS is currently only implemented in C++, python and Java (Ferguson). Fortunately Patrick Goebel wrote a python version (Goebel, 2014). Ferguson's library is able to be used on a Raspberry Pi, and documentation is English, so it was chosen over the other options.
\\
TODO: discuss bug fixes done?
\\
\\
There are several main methods of steering, each with their own advantages and disadvantages (Shamah et. al., 2001). By virtue of chassis design a range of steering options were unable to be implemented. These included ackerman, independent, synchronous or omnidirectional steering. The remaining options were skid steering and articulated drive.
\\
\\
Skid steering has very high maneuverability for low mechanical complexity. If skid steering alone is implemented, a lot of space is saved because of the low number of components. However because of the differential thrust from the right and left there is relatively poor traction, and there is relatively high wear on the wheels. This leads to a relatively high power consumption. (Kang et. al., 2010)
\\
\\
Conversely articulated drive has a significantly larger turning circle, and  maintains traction throughout the turn, allowing for acceleration throughout. Implementation is more involved than skid steering however, as outer wheels have to be turning faster than inner wheels. Shamah et. al. (2001) developed a series of velocity calculation equations for a four wheeled vehicle with an articulated axle. The velocity of the individual wheel was calculated based on the distance from each individual wheel to the centre of the circle, and the desired robot velocity. (See fig 1.)
\\
\\
As a library was being made for use with the OARKit, both skid steering and articulated steering were implemented. This would allow the end user to choose the most suitable method for their environment and design specification.
\\
\\
\begin{figure}
\includegraphics[width=\textwidth]{fig1}
\caption{(Shamah et. al. 2001)}
\end{figure}
Used ROS joystick driver to teleop and testing and blah (ref).

\section{Networking and Interface}
It is crucial for the robot to be able to communicate over the network in some form, so that it can operate without being tethered to cables. The onboard Raspberry Pi and dynamixel motors are equipped with portable power sources, and a wifi dongle allows the Pi to communicate over a network over wifi.

\subsection{A Sample Interface}
We designed a sample interface to use the high level API functions provided by our pyOARKit library. Contained in interface.py, it can be run from the terminal and accepts input from the keyboard.
\\
\\
The interface is very simple and involves text-based commands of the form <command><value> where <command> is a single letter and <value> is typically an integer or a float.
\\
\\
\begin{tabular}{ | l | l | p{5cm} | }
\hline
\textbf{Syntax} & \textbf{Value Range} & \textbf{Description} \\ \hline
v<velocity> & Integer from -1023 to 1023 & Sets the velocity - moves the robot forwards and backwards. \\ \hline
s<velocity> & Integer from -1023 to 1023 & Skid steering - a positive value moves the right wheels forward and the rear wheels backwards. Can be used in conjunction with velocity setting. \\ \hline
t<velocity> & Integer from -1023 to 1023 & Turn at the given velocity - a positive value turns left. \\ \hline
l<velocity> & Integer from -1023 to 1023 & Lift the front segment up at the given velocity - a positive value lifts upwards \\ \hline
T<angle> & Float from -1 to 1 & Turn to the given normalised angle at an arbitrary speed of 50. 1 is the extreme left and -1 is the extreme right while 0 is the centre position. \\ \hline
L<angle> & Float from -1 to 1 & Lift to the given normalised angle at an arbitrary speed of 50. 1 is the maximum height and -1 is the minimum height while 0 is the centre position - flat. \\ \hline
\end{tabular}

\subsection{ROS Integration}
We wanted integration with ROS in order to leverage the existing ecosystem of software and tools. Initially, the intention was to install ROS on the onboard Raspberry Pi but this proved to be complicated since ROS only officially targets Ubuntu. Running ROS as well as controlling the nine dynamixel motors on the Pi might also be difficult since the Pi (Model B+) only boasts 512mb of RAM and a 700MHz CPU.
\\
\\
Instead, we decided to run a small UDP server on the Pi and run the ROS node on a desktop/laptop.

\subsubsection{server.py}
This is a small UDP server run on the Raspberry Pi and uses the sample interface.py. It accepts UDP messages which contain commands in the same format as the sample interface. From the Pi:
\begin{lstlisting}[language=bash]
  $ python server.py 9999 # Run the server on port 9999
\end{lstlisting}

\subsubsection{ROS Node}
Our sample ROS node runs on a desktop or laptop and sends UDP messages to the server on the Pi based on input from a joystick. We used a logitech controller in our demo, but the code should be able to be easily adapted for other joysticks or similar teleop nodes. Ensure that the joystick is plugged in, and then:
\begin{lstlisting}[language=bash]
  $ sudo apt-get install ros-indigo-joystick-drivers # Make sure you have the joystick drivers
  $ roslaunch logitech.launch

  $ cd workspace
  $ catkin_make
  $ rosrun mantis move.py <ip_of_raspberry_pi> 9999 # Connect to the Pi on port 9999
\end{lstlisting}

\subsubsection{Logitech Joystick}
You should now be able to control the robot via the Logitech joystick. The control scheme is as follows:
\\
\\
\begin{tabular}{ | l | p{7cm} | }
\hline
\textbf{Control} & \textbf{Description} \\ \hline
Left joystick up and down & Moves the robot forwards and backwards. \\ \hline
Left joystick left and right & Performs skid steering on the robot. \\ \hline
Right joystick up and down & Raise and lower the front segment of the robot. \\ \hline
Right joystick left and right & Turn the robot left and right. \\ \hline
Hold Start button & Return the robot to the neutral position - turn to face forwards and raise/lower the front segment to be parallel to the ground. \\ \hline
\end{tabular}

\subsection{Design Decisions and Potential Issues}
The main decision to make was to decide on the protocol used to send messages from the ROS node to the server on the Pi. We chose to use UDP over TCP or any other protocols because being able to control the robot in real time was a priority. There were potentially a lot of messages being sent very quickly when moving the joystick and using TCP could have caused some unwanted delay.
\\
\\
We weren't too worried about dropped packets (UDP does not guarantee delivery) because our interface and library are mostly stateless. That is, subsequent commands are mutually exclusive and do not affect one another. At worst, it would just require inputting a particular command again. However, we did run into an issue where it was repeatedly observed that actions on the joystick sometimes would not register. We pinpointed the issue to a few potential sources:
\begin{enumerate}
  \item Dropping packets over the network
  \item Congestion in the dynamixel network causing commands sent to the dynamixels to not take effect
  \item Incorrect report from the joystick itself
\end{enumerate}
We tried changing our server and client to use TCP instead of UDP but the issue persisted. It also seems unlikely that there would be a hardware issue with the joystick or the joystick drivers but we were unable to verify this. A possible solution could be to try lowering the rate at which commands are sent to the dynamixels (currently every 0.2 seconds). However, this would likely introduce noticeable input delay.


\section{References}
Bigge, R. (2011). \textit{Robots to the rescue}. Available at\\
https://secure.globeadvisor.com/servlet/ArticleNews/story/gam/20110527\\
/ROBMAG\_JUNE2011\_P12\_14\_ (accessed 04/11/2014).\\
\\
Davids, A. (2002). Urban search and rescue robots: from tragedy to technology. \textit{Intelligent Systems, IEEE 17}(2). 81 - 83.\\
\\
Gindel, P. (2010). \textit{Arduino library for AX-12}. \\Available at http://www.pablogindel.com/2010/01/biblioteca-de-arduino-para-ax-12/. (accessed 09/11/14).\\
\\
Kang, J., Kim W., Lee, J., \& Yi, K. (2010). Design, implementation, and test of skid steering-based autonomous driving controller for a robotic vehicle with articulated suspension. \textit{Journal of Mechanical Science and Technology 24}(3). 793 - 800.\\
\\
Kobe University Library. (1999) \textit{Great Hanshin-Awaji Earthquake Disaster Materials Collection.} Available at: http://www.lib.kobe-u.ac.jp/eqb/e-gallery.html (accessed 04/11/2014).\\
\\
The Open Academic Robotic Kit. Available at www.oarkit.org (accessed 09/11/2014).\\
\\
Shamah, B.,  Wagner, M. D., Moorehead S., Teza, J., Wettergreen, D., \& Whittaker, W. (2001). \textit{Steering and Control of a Passively Articulated Robot}. The Field Robotics Center, Carnegie Mellon University.\\
\\
Sheh, R. (2014). \textit{Open Academic Robot Kit: The Six-Wheeled Wonder - a 6 Wheel Drive robot platform using Dynamixel AX-12A servos}. Available at: http://www.thingiverse.com/thing:327689 (accessed 04/11/2014).

\end{document}
